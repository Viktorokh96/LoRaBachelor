\Introduction

С ростом информационных технологий и коммуникаций в XXI веке новые тенденции проявляются в виде интеллектуальной обработки больших массивов данных (Big Data) и Интернете вещей (IoT или IdC). 
Интернет не только позволил объединить людей, но также и устройства, датчики в сложные системы. 
Так появляется термин «Интернет вещей», который соответствует соединению различных устройств с Интернетом, генерирующие данные в реальном времени. 
Так называемые сети с низким энергопотреблением и большим покрытием (LPWA) являются мостом к Интернету Вещей, разработанные с целью заполнить брешь в технологиях с низким энергопотреблением, большим покрытием и низкой стоимостью. 
Одной из лидирующих технологии LPWAN является LoRaWAN. Она и станет объектом рассмотрения данной работы.

Целями данной работы являются:

\begin{itemize}
	\item изучить возможные применения технологии в контексте развития Интернета вещей;
	\item сравнить открытый стандарт LoRaWAN с другими технологиями беспроводной передачи данных, таких как SigFox, LTE, Стриж и т.д;
	\item оценить зону покрытия в городской среде для 
радиоприёмника LoRa SX1278 в разных режимах работы;
	\item создать программное обеспечение для работы с SX1278 в связке с
STM32L476, используя инструменты открытого программного обеспечения;
	\item задокументировать полученное программное обеспечение.
\end{itemize}

\Abbrev{LPWAN}{Low-power Wide-aread Network ""--- энергоэффективная сеть дальнего радиуса действия}
\Abbrev{LoRa}{Long Range ""--- технология энергоэффективной беспроводной связи дальнего действия}
\Abbrev{UL}{Up-link ""--- восходящий канал связи}
\Abbrev{DL}{Down-link ""--- нисходящий канал связи}
\Abbrev{CSS}{Chirp spread spectrum ""--- линейная частотная модуляция}
\Abbrev{ISM}{industrial, scientific and medical radio bands""--- международные зарезервированные частотные дипазоны}
\Abbrev{NGN}{Next Generation Network ""--- сеть последующих поколений}
\Abbrev{IoT}{с англ. \textit{Internet of Things} в переводе ""--- \textit{``Интернет вещей''}} 
\Abbrev{LTE}{Long Term Evolution ""--- долгосрочное развитие}
%\Abbrev{API}{Application Programming Interface ""--- внешний 
%интерфейс взаимодействия с приложением}

\Define{Интернет}{глобальная вычислительная сеть. Является самой большой 
компьютерной сетью в мире. Построена на базе стека протоколов TCP/IP}
\Define{Трансивер}{Приёмопередатчик}
\Define{Сетевой протокол}{соглашение о наборе правил, повзоляющих проводить 
соединение и обмен данными между двумя и более устройствами, подключенным к 
сети}
\Define{Сеть последующих поколений (СПП) (Next generation network)}{Сеть с пакетной коммутацией, пригодная для предоставления услуг электросвязи и для использования нескольких широкополосных технологий транспортировки с ключенной функцией QoS, в который связанные с обслуживанием функции не зависят от примененных технологий, обепечивающих транспортировку}
\Define{Чирп}{(от англ. chirp) ""--- отдельный сигнал, заданный на всей полосе частот канала (существует схожий по названию термин из теории обработки сигналов - ``чирплет'')}


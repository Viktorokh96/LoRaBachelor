% Также можно использовать \Referat, как в оригинале
\begin{abstract}

    Отчет содержит \pageref{LastPage}\,стр.%
    \ifnum \totfig >0
    , \totfig~рис.%
    \fi
    \ifnum \tottab >0
    , \tottab~табл.%
    \fi
    %
    \ifnum \totbib =1
    , \totbib~источник%
    \else
    \ifnum \totbib >1
    \ifnum \totbib <5
    , \totbib~источника%
    \else
    , \totbib~источников%
    \fi
    \fi
    \fi
    %
    \ifnum \totapp >0
    , \totapp~прил.%
    \fi

	Данная работа ставит перед собой цель изучения технологии беспроводной передачи данных LoRa в контексте её использования в Интернете вещей.

	В работе кратко отображено современное состояние Интернета вещей и беспроводных технологии передачи небольших по объёму данных на дальние расстояние и с низким энергопотреблением.
	Рассмотрен способ использования принципа открытого программного обеспечения для создания конечных устройств в сети LoRaWAN.

	Ключевые слов: LoRa, Интернет вещей, беспроводные технологии передачи данных, открытое программное обеспечение, разработка ПО.

    \nocite{*}

\end{abstract}

%%% Local Variables: 
%%% mode: latex
%%% TeX-master: "rpz"
%%% End: 

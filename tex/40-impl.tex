\chapter{Технологический раздел}
\label{cha:impl}

В данном разделе описано изготовление и требование всячины. Кстати,
в Latex нужно эскейпить подчёркивание (писать <<\verb|some\_function|>> для \Code{some\_function}).

\ifPDFTeX
Для вставки кода есть пакет \Code{listings}. К сожалению, пакет \Code{listings} всё ещё
работает криво при появлении в листинге русских букв и кодировке исходников utf-8.
В данном примере он (увы) на лету конвертируется в koi-8 в ходе сборки pdf.

Есть альтернатива \Code{listingsutf8}, однако она работает лишь с
\Code{\textbackslash{}lstinputlisting}, но не с окружением \Code{\textbackslash{}lstlisting}

Вот так можно вставлять псевдокод (питоноподобный язык определен в \Code{listings.inc.tex}):

\begin{lstlisting}[style=pseudocode,caption={Алгоритм оценки дипломных работ}]
def EvaluateDiplomas():
    for each student in Masters:
        student.Mark := 5
    for each student in Engineers:
        if Good(student):
            student.Mark := 5
        else:
            student.Mark := 4
\end{lstlisting}

Еще в шаблоне определен псевдоязык для BNF:

\begin{lstlisting}[style=grammar,basicstyle=\small,caption={Грамматика}]
  ifstmt -> "if" "(" expression ")" stmt |
            "if" "(" expression ")" stmt1 "else" stmt2
  number -> digit digit*
\end{lstlisting}

В листинге~\ref{lst:sample01} работают русские буквы. Сильная магия. Однако, работает
только во включаемых файлах, прямо в \TeX{} нельзя.

% Обратите внимание, что включается не ../src/..., а inc/src/...
% В Makefile есть соответствующее правило для inc/src/*,
% которое копирует исходные файлы из ../src и конвертирует из UTF-8 в KOI8-R.
% Кстати, поэтому использовать можно только русские буквы и ASCII,
% весь остальной UTF-8 вроде CJK и египетских иероглифов -- нельзя.

\lstinputlisting[language=C,caption=Пример (\Code{test.c}),label=lst:sample01]{inc/src/test.c}

\else

Для вставки кода есть пакет \texttt{minted}. Он хорош всем кроме: необходимости Python (есть во всех нормальных (нет, Windows, я не про тебя) ОС) и Pygments и того, что нормально работает лишь в \XeLaTeX.

\ifdefined\NoMinted
Но к сожалению, у вас, по-видимому, не установлен Python или pygmentize.
\else
Можно пользоваться расширенным BFN:

\begin{listing}[H]
\begin{ebnfcode}
 letter = "A" | "B" | "C" | "D" | "E" | "F" | "G"
       | "H" | "I" | "J" | "K" | "L" | "M" | "N"
       | "O" | "P" | "Q" | "R" | "S" | "T" | "U"
       | "V" | "W" | "X" | "Y" | "Z" ;
digit = "0" | "1" | "2" | "3" | "4" | "5" | "6" | "7" | "8" | "9" ;
symbol = "[" | "]" | "{" | "}" | "(" | ")" | "<" | ">"
       | "'" | '"' | "=" | "|" | "." | "," | ";" ;
character = letter | digit | symbol | "_" ;
 
identifier = letter , { letter | digit | "_" } ;
terminal = "'" , character , { character } , "'" 
         | '"' , character , { character } , '"' ;
 
lhs = identifier ;
rhs = identifier
     | terminal
     | "[" , rhs , "]"
     | "{" , rhs , "}"
     | "(" , rhs , ")"
     | rhs , "|" , rhs
     | rhs , "," , rhs ;
 
rule = lhs , "=" , rhs , ";" ;
grammar = { rule } ;
\end{ebnfcode}
\caption{EBNF определённый через EBNF}
\label{lst:ebnf}
\end{listing}

А вот в листинге \ref{lst:c} на языке C работают русские комменты. Спасибо Pygments и Minted за это.

\begin{listing}[H]
\cfile{inc/src/test.c}
\caption{Пример — test.c} 
\end{listing}
\label{lst:c}

\fi
\fi
% Для вставки реального кода лучше использовать \texttt{\textbackslash lstinputlisting} (который понимает
% UTF8) и стили \Code{realcode} либо \Code{simplecode} (в зависимости от размера куска).




Можно также использовать окружение \Code{verbatim}, если \Code{listings} чем-то не
устраивает. Только следует помнить, что табы в нём <<съедаются>>. Существует так же команда \Code{\textbackslash{}verbatiminput} для вставки файла.

\begin{verbatim}
a_b = a + b; // русский комментарий
if (a_b > 0)
    a_b = 0;
\end{verbatim}

%%% Local Variables:
%%% mode: latex
%%% TeX-master: "rpz"
%%% End:

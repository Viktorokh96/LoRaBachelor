\Introduction

С ростом информационных технологий и коммуникаций в XXI веке новые тенденции проявляются в виде интеллектуальной обработки больших массивов данных (Big Data) и Интернете вещей (IoT или IdC). 
Интернет не только позволил объединить людей, но также и устройства, датчики в сложные системы. 
Так появляется термин «Интернет вещей», который соответствует соединению различных устройств с Интернетом, генерирующие данные в реальном времени. 
Так называемые сети с низким энергопотреблением и большим покрытием (LPWA) являются мостом к Интернету Вещей, разработанные с целью заполнить брешь в технологиях с низким энергопотреблением, большим покрытием и низкой стоимостью. 
Одной из лидирующих технологии LPWAN является LoRaWAN. Она и станет объектом рассмотрения данной работы.

Целями данной работы являются:


\begin{itemize}
	\item изучить возможные применения технологии в контексте развития Интернета Вещей;
	\item сравнить открытый стандарт LoRaWAN с другими технологиями беспроводной передачи данных, таких как SigFox, LTE, Стриж и т.д;
	\item оценить характеристики энергопотребления и зону покрытия в городской среде для радиоприёмника LoRa SX1278 в разных режимах работы;
	\item создать и задокументировать программное обеспечение для работы с SX1278 в связке с STM32L476, используя инструменты открытого программного обеспечения.
\end{itemize}

\Abbrev{LTE}{LTE, Long Term Evolution ""--- долгосрочное развитие}
\Abbrev{NGN}{Next Generation Network ""--- сеть последующих поколений}
\Abbrev{IoT}{с англ. \textit{Internet of Things} в переводе ""--- \textit{``Интернет вещей''}} 
\Abbrev{API}{application programming interface ""--- внешний интерфейс взаимодействия с приложением}
\Define{Интернет}{глобальная вычислительная сеть. Является самой большой компьютерной сетью в мире. Построена на базе стека протоколов TCP/IP.}
\Define{Сетевой протокол}{соглашение о наборе правил, повзоляющих проводить соединение и обмен данными между двумя и более устройствами, подключенным к сети}

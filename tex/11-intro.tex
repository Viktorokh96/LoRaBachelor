\Defines

\begin{itemize}
 \item[] Интернет ""--- глобальная вычислительная сеть. Является самой 
большой компьютерной сетью в мире. Построена на базе стека протоколов TCP/IP.
 \item[] Трансивер ""--- приёмопередатчик.
 \item[] Сетевой протокол ""--- соглашение о наборе правил, позволяющих 
проводить соединение и обмен данными между двумя и более устройствами, 
подключенным к сети.
 \item[] Сеть последующих поколений (СПП) (Next generation network) ""--- Сеть 
с пакетной коммутацией, пригодная для предоставления услуг электросвязи и для 
использования нескольких широкополосных технологий транспортировки с 
подключенной функцией QoS, в который связанные с обслуживанием функции не 
зависят от примененных технологий, обеспечивающих транспортировку.
\end{itemize}

\Abbreviations
\begin{itemize}
  \item[] LPWAN ""--- Low-power Wide-aread Network ""--- энергоэффективная сеть 
дальнего радиуса действия.
  \item[] LoRa\texttrademark ""--- Long Range ""--- технология энергоэффективной 
беспроводной 
связи дальнего действия.
  \item[] UL ""--- Up-link восходящий канал связи.
  \item[] DL ""--- Down-link ""--- нисходящий канал связи.
  \item[] CSS ""--- Chirp spread spectrum ""--- линейная частотная модуляция.
  \item[] ISM ""--- industrial, scientific and medical radio bands""--- 
международные зарезервированные частотные диапазоны.
  \item[] NGN ""--- Next Generation Network ""--- сеть последующих поколений.
  \item[] IoT ""--- с англ. \textit{Internet of Things} в переводе ""--- 
\textit{``Интернет вещей''}.
\end{itemize}

\Introduction
С ростом информационных технологий и коммуникаций в XXI веке новые тенденции 
проявляются в виде интеллектуальной обработки больших массивов данных (Big Data) 
и Интернете вещей (IoT или IdC). 
Интернет не только позволил объединить людей, но также и устройства, датчики в 
сложные системы. 
Так появляется термин «Интернет вещей», который соответствует соединению 
различных устройств с Интернетом, генерирующие данные в реальном времени. 
Так называемые сети с низким энергопотреблением и большим покрытием (LPWA) 
являются мостом к Интернету Вещей, разработанные с целью заполнить брешь в 
технологиях с низким энергопотреблением, большим покрытием и низкой стоимостью. 
Одной из лидирующих технологии LPWAN является LoRaWAN\texttrademark. Она и 
станет объектом рассмотрения данной работы.

Целью данной работы является исследование технологии LoRa\texttrademark и 
оценка применимости 
данной технологии для инфраструктуры Интернета вещей.

Поставленные задачи:
\begin{itemize}
	\item изучить возможные применения технологии LoRa\texttrademark~ в
Интернете вещей;
	\item сравнить открытый стандарт LoRaWAN\texttrademark с другой 
технологией
беспроводной передачи данных, такой как SigFox;
	\item оценить зону покрытия в городской среде для 
радиоприёмника LoRa\texttrademark SX1278 в разных режимах работы;
	\item создать программное обеспечение для работы с SX1278 в связке с
STM32L476, используя инструменты открытого программного обеспечения;
	\item задокументировать полученное программное обеспечение.
\end{itemize}


%\Abbrev{API}{Application Programming Interface ""--- внешний 
%интерфейс взаимодействия с приложением}


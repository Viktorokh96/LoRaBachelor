\chapter{Аналитический обзор}
\label{cha:analysis}
%
% % В начале раздела  можно напомнить его цель
%
%В данном разделе анализируется и классифицируется существующая всячина и пути создания новой всячины. А вот отступ справа в 1 см. "--- это хоть и по ГОСТ, но ведь диагноз же...

%%%%%%%%%%%%%%%%%%%%%%%%%%%%%%%%%%%%%%%%%%%%%%%%%%%%%%%%%%%%%%%%%%%%%%%%%%%%%%%%%%%%%%%%%%%%%%%%%%%%%%%%%%%%%%%%

\section{Что такое Интернет вещей}

Интернет вещей (англ. \textit{IoT}, \textit{Internet of Things}) ""--- это методология вычислительной сети физических объектов (``\textit{вещей}''), имеющих встроенную поддержку технологий передачи данных для их взаимодействия, а также для взаимодействия с внешней средой.
Эта методология рассматривает Интернет вещей как явление, способное перестроить культурные и экономические процессы, всё больше исключая человека из них.
Влияние существующего Интернета на сферы образования, коммуникации, бизнеса, науки и политики позволяет говорить о том, что Интернет является одним из важнейших и мощнейших изобретений в истории человечества \cite{evans2011internet}.
Интернет вещей стоит рассматривать как новую ветвь эволюции Интернета, где каждый предмет в поле зрения человека может быть оснащён датчиками, сенсорами, устройством управления и модулем передачи данных для общения со всем миром.

Как известно, большинство великих изобретений человечества потребовали десятки и даже сотни лет на переход от простых по форме представлений до сложных систем.
От создания предпосылок, до массового внедрения Интернета ушло почти четверть века, однако похоже что для Интернета вещей на то же самое потребуется существенно меньше времени \cite{chernyak2013}.
Международный союз связи (МСЭ) и Европейский Союз определили Интернету вещей 
главенствующую роль в дальнейшем развитии информационных технологий. 
По расчетам консалтингового подразделения Cisco IBSG (см. рис. \ref{fig:iotandpeople}) в промежутке между 2008 и 2009 годами, количество устройств, подключенных к интернету, превысило количество людей, и к 2020 году количество подключенных устройств достигнет 50 миллиардов \cite{evans2011internet} (по другим данным \cite{denise2014} ""--- 25 миллиардов).
Таким образом, в настоящее время происходит переход от ``Интернета людей'' к ``Интернету вещей''.
Хотя данная концепция на международном уровне уже обретает черты сформировавшейся технологии, для неё ведутся активные работы в области стандартизации компонентов, архитектуры и приложений.
Количество мнений о том как будет построен Интернет вещей очень велико. 
Это подтверждается большим разнообразием предлагаемых технологий для создания LPWAN сетей на рынке.

\begin{figure}
  \centering
  \includegraphics[width=0.8\textwidth]{inc/img/IoTAndPeople.pdf}
	\caption{Временная шкала изменения количества людей и предметов, 
подключенных к интернету}
  \label{fig:iotandpeople}
\end{figure}

%Кстати, про картинки. Во-первых, для фигур следует использовать \texttt{[ht]}. Если и после этого картинки вставляются <<не по ГОСТ>>, т.е. слишком далеко от места ссылки, "--- значит у вас в РПЗ \textbf{слишком мало текста}! Хотя и ужасный параметр \texttt{!ht} у окружения \texttt{figure} тоже никто не отменял, только при его использовании документ получается страшный, как в ворде, поэтому просьба так не делать по возможности.

\subsection{Базовые принципы Интернета вещей}

Интернет вещей основывается на трёх базовых принципах \cite{roslyakov2014}.
\begin{enumerate}
	\item повсеместно распространенная инфраструктура;
	\item глобальная идентификация каждого объекта;
	\item возможность каждого физического объекта отправлять и получать данные, посредством локальной сети или сети Интернет, к которой он подключен.
\end{enumerate}

Наиболее важными отличиями Интернета вещей от интернета людей являются:
\begin{itemize}
	\item фокус на считывание информации, а не на коммуникациях;
	\item на порядки большее число подключенных к сети объектов;
	\item потребность в создании новых стандартов;
	\item намного меньше размеры объектов и скорости передачи данных;
	\item фокус не на человеке, а на вещах;
\end{itemize}

Концепция сетей следующего поколения NGN предполагала возможность коммуникаций людей в любой точке пространства и времени.
Концепция интернета вещей включает ещё одно направление ""--- коммуникация любых вещей или устройств (рис. \ref{fig:iotconcept})

\begin{figure}
  \centering
  \includegraphics[width=0.8\textwidth]{inc/img/IoTConcept.pdf}
	\caption{Новое направление коммуникаций, реализуемой Интернетом вещей}
  \label{fig:iotconcept}
\end{figure}

Согласно принятым в МСЭ-Т представлениям о отображении физических и виртуальных 
вещей, виртуальные вещи могут обходиться без их физического соответствия, в то 
время как каждой физической вещи соответствует минимум один объект в виртуальном 
пространстве (см. рис. \ref{fig:physvirtworld}).  

Рекомендация Y.2060 от МСЭ-Т описывает различное сочетание способов соединений.
МСЭ-Т рассматривает множество сетевых технологий, как потенциально пригодных для приложений Интернета вещей, а именно: глобальные сети, локальные сети, ячеистые (mesh) сети и беспроводные самоорганизующиеся (ad-hoc) сети.

\begin{figure}
  \centering
  \includegraphics[width=0.9\textwidth]{inc/img/realvirtualthings.pdf}
	\caption{Отображение физических и виртуальных вещей}
  \label{fig:physvirtworld}
\end{figure}

\subsection{Основные характеристики Интернета вещей}

IoT, имеет следующие характеристики:

\begin{itemize}
	\item возможность установления соединений. Любую вещь можно соединить к Интернету вещей;
	\item гетерогенность: устройства в концепции Интернета вещей являются гетерогенными и базируются на различных аппаратных платформах и сетях. 
		Могут обмениваться информацией с другими устройствами, независимо от структуры сети и применяемых технологий транспортного уровня. 
		Примечательно, что современное состояние сети Интернет может удовлетворить этому лишь отчасти: пул адресов IPv4 исчерпан и большая часть устройств скрывается в локальных сетях за устройствами NAT, что противоречит изначальной концепции однородного интернета. 
		Решением может стать повсеместное использование протокола IPv6 в качестве протокола сетевого уровня. 
		Хотя внедрение этого протокола и затянулось, но уже к декабрю 2018 года ожидается, что 25\% всех Интернет доменов будет доступно через этот протокол \cite{pickard2017}.;
	\item огромное количество одновременно подключенных устройств к сети, 
которыми необходимо управлять, обмениваться данными. Произойдёт существенное 
увеличение долей обмена данными, инициированными устройствами, по сравнению с 
долей информационного обмена, инициированного людьми;
	\item динамические изменения структуры сети. Устройства будут свободно подключаться к сетям, менять своё местоположение, отключаться от сети и подключаться к новым устройствам. Подразумевается, что количество устройств в одной сети - переменная величина с течением времени;
\end{itemize}

Важной частью рекомендации от МСЭ-Т являются требования	предъявляемые к 
устройствам IoT. Любые технологии LPWAN, в том числе и LoRa\texttrademark, 
должны соответствовать этим требованиям для предоставления возможности их 
включения в инфраструктуру Интернета вещей:

\begin{itemize}
	\item предоставление автономных услуг: требуется, чтобы услуги могли 
предоставляться с помощью автоматической передачи, обработки и сбора данных 
вещей, основанных на правилах, задаваемых операторами или абонентами. Услуги 
могут зависеть от методов автоматизированной обработки и интеллектуального 
анализа данных;
	\item соединение на основе идентификатора: соединение с любой вещью в 
концепции Интернета вещей будет происходить на основе уникального 
идентификатора, которым обладает тот или иной объект. Отсюда выходит требование 
о создании универсального идентификатора (например адрес IPv6) для применения в 
гетерогенных сетях. \item функциональная совместимость: требуется обеспечение 
функциональной совместимости гетерогенных и распределенных систем в целях 
предоставления и потребления самый разных видов услуг;
	\item  возможности определения местоположения: требуется, чтобы в 
Интернете вещей обеспечивались услуги, на основе информации о местоположении 
объекта. Требуется, чтобы информация о местоположении вещей отслеживалась 
автоматически. Связь и услуги на основе местоположения могут быть ограничены 
законами и нормативными актами и должны соответствовать требованиям 
безопасности; \item безопасность в Интернете вещей: каждая вещь имеет 
соединение 
с сетью, что приводит к серьёзной угрозе безопасности, таким как угроза 
аутентичности, целостности и конфиденциальности как данных, так и услуг. Одним 
и 
важнейших требований к безопасности является необходимость объединения 
различных 
методов и принципов обеспечения безопасности множества устройств и сетей 
пользователей;
	\item защита неприкосновенности частной жизни: требуется, чтобы в IoT 
обеспечивалась неприкосновенность частной жизни. У многих вещей есть владельцы 
и 
эти вещи могут хранить личную информацию их владельцев. Необходимо обеспечить 
неприкосновенность частной жизни человека при сборе, обработке, анализе и 
передачи больших массивов информации вещами. Защита неприкосновенности частной 
жизни не должна служить препятствием для аутентификации источника данных; \item 
автоматическое конфигурирование: необходимо обеспечить возможность 
автоматического конфигурирования устройств, для возможности оперативной 
модификации программного обеспечения вещей, с целю повысить качество 
обслуживания клиентов, а также степень интеграции устройства с окружающим миром 
и сетью, не нарушая при этом, требования о безопасности и конфиденциальности.
	\item управляемость: возможность вмешательства человека в работу вещей 
при необходимости. 
\end{itemize}

\subsection{Эталонная модель Интернета вещей}

Также была разработана эталонная модель интернета вещей \cite{itutiot2012}, она 
показана на рисунке \ref{fig:iotetalon}.
Она включает в себя четыре уровня, а также возможности обеспечения безопасности 
и управления, которые связаны с этими четырьмя уровнями:

\begin{itemize}
	\item уровень приложения;
	\item уровень поддержки услуг и поддержки приложения;
	\item уровень сети;
	\item уровень устройства.
\end{itemize}

\begin{figure}
  \centering
  \includegraphics[width=\textwidth]{inc/img/IoTEtalon.pdf}
	\caption{Эталонная модель IoT}
  \label{fig:iotetalon}
\end{figure}

%%%%%%%%%%%%%%%%%%%%%%%%%%%
%%%%%%%%%%%%%%%%%%%%%%%%%%%

\subsubsection{Уровень приложения}

Содержит само приложение IoT.

%%%%%%%%%%%%%%%%%%%%%%%%%%%
%%%%%%%%%%%%%%%%%%%%%%%%%%%

\subsubsection{Уровень поддержки услуг и поддержки приложений}

Данный уровень состоит из следующих двух групп возможностей:

\begin{itemize}
	\item общие возможности поддержки, или типовые возможности, которые 
могут использоваться приложениями Интернета вещей такими, как хранение или 
обработка данных.
	\item специализированные возможности поддержки или набор конкретных 
возможностей, предназначенных для удовлетворения требований разнообразных 
приложений.
\end{itemize}


%%%%%%%%%%%%%%%%%%%%%%%%%%%
%%%%%%%%%%%%%%%%%%%%%%%%%%%

\subsubsection{Уровень сети}

Существует два типа возможностей:

\begin{itemize}
	\item возможности организации сетей: предоставляет функции управления 
сетевыми соединениями;
	\item возможности транспортировки: предназначены для предоставления 
соединений для транспортировки информации в виде данных, относящихся к услугам и 
приложениям IoT, а также транспортировки информации управления и контроля, 
относящейся к IoT.
\end{itemize}

%%%%%%%%%%%%%%%%%%%%%%%%%%%
%%%%%%%%%%%%%%%%%%%%%%%%%%%

\subsubsection{Уровень устройства}

Этот уровень можно логически разделить на два вида возможностей:
\begin{itemize}
	\item возможности устройства. Это могут быть такие возможности, как: 
спящий режим и пробуждение, организация специальных сетей, прямое и непрямое 
взаимодействие устройства с сетью;
	\item возможности шлюза. Это возможность поддержки различных 
интерфейсов. Шлюза объединяют в себе различные сетевые интерфейсы, как 
проводные, так и беспроводные.
\end{itemize}

%%%%%%%%%%%%%%%%%%%%%%%%%%%
%%%%%%%%%%%%%%%%%%%%%%%%%%%

\subsubsection{Возможности управления}

Возможности управления IoT охватывают традиционные классы конфигурации, учета, 
безопасности и т.д.

Важнейшие возможности управления включают:
\begin{itemize}
	\item управление устройствами, диагностика, обновление, прошивка, 
управление рабочим состоянием устройства;
	\item управление топологией локальной сети;
	\item управление трафиком и перегрузками.
\end{itemize}

\subsubsection{Возможности обеспечения безопасности}

Есть два вида возможностей обеспечения безопасности: общие и специализированные.
Общие возможности не зависят от приложений и включают:
\begin{itemize}
	\item на уровне приложений: авторизацию, конфиденциальность, 
аутентификацию, целостность данных приложения, защиту неприкосновенности частной 
жизни, аудит безопасности;
	\item на уровне сети: авторизацию, аутентификацию, защиту 
конфиденциальности и целостности;
	\item на уровне устройства: аутентификацию, авторизацию, проверку 
целостности устройства, управление доступом, защиту целостности и 
конфиденциальности.
\end{itemize}

Специализированные возможности зависят от вида приложений и могут налагать 
дополнительные специфичные требования по безопасности.

%%%%%%%%%%%%%%%%%%%%%%%%%%%%%%%%%%%%%%%%%%%%%%%%%%%%%%%%%%%%%%%%%%%%%%%%%%%%%%%%
%%%%%%%%%%%%%%%%%%%%%%%%%%%%%%%%
\newpage
\section{Обзор технологии LoRa\texttrademark} 

LoRa\texttrademark~представляет собой технологию энергоэффективной сети 
дальнего радиуса действия, разрабатываемый организацией 
LoRa\texttrademark~Alliance.
Данная технология нацелена на использование в устройствах с автономными 
источниками питания, где показатель энергопотребления является наиболее важным.
В данном разделе будет дан обзор на данную технологию.
Будут кратко рассмотрены физический и уровень управления доступом к сети (MAC) 
LoRaWAN\texttrademark~сетей.

\subsection{Стек протоколов LoRa\texttrademark}

На рисунке \ref{fig:lorastack} изображён стек протоколов в сетях 
LoRaWAN\texttrademark. 

\begin{figure}[!h]
  \centering
  \includegraphics[width=0.9\textwidth]{inc/img/LoRaStack.pdf}
	\caption{Стек протоколов LoRaWAN\texttrademark}
  \label{fig:lorastack}
\end{figure}

Далее будут кратко рассмотрены все уровни данного стека протоколов.

\subsection{Сетевая архитектура LoRa\texttrademark}

Стандартной топологией LoRa\texttrademark~является ``звезда из звёзд'', которая 
включает в 
себя различные типы устройств, как показано на рисунке 
\ref{fig:loranetworkarch}.

\begin{figure}[ht]
  \centering
  \includegraphics[width=\textwidth]{inc/img/LoRaNetworkArch.pdf}
  \caption{Сетевая архитектура LoRa\texttrademark}
  \label{fig:loranetworkarch}
\end{figure}

%%%%%%%%%%%%%%%%%%%%%%%%%%%
\subsection{Физический уровень}

Технология LoRa\texttrademark~описывает два независимых уровня протоколов: 
физический, с 
использование линейной частотной модуляцией (CSS); и протокол контроля доступа 
к 
среде (LoRaWAN\texttrademark), хотя системы коммуникации LoRa\texttrademark~
также реализуют 
специфичные сетевые архитектуры \cite{augustin2016}.

Физический уровень LoRa\texttrademark~разработан компанией Semtech и он 
обеспечивает связь с 
низким энергопотреблением, низкой скоростью и большим радиусом действия.
Размер полезных данных может изменятся в диапазоне от 2 до 255 октетов, а 
скорость передачи данных может достигать до 50 Кбит/c. 
Технология модуляции закрыта и является собственностью компании Semtech, 
поэтому 
здесь будут рассмотрены только известные принципы работы передатчиков и 
приёмников LoRa\texttrademark.

LoRa\texttrademark~использует линейную частотную модуляцию, которая использует 
линейное 
изменение частоты несущей во время передачи закодированной информации.
Благодаря линейному возрастанию частоты сигнала смещение частоты между 
приёмником и передатчиком за промежуток времени передачи символа остаётся 
постоянным, что легко устраняется демодулятором \cite{augustin2016}. 
Это делает данную модуляцию невосприимчивой к эффекту Доплера.
Смещение частот между приёмником и передатчиком может достигать 20\% ширины 
полосы частот без влияния на корректность демодуляции.
Это помогает уменьшить стоимость приёмников и передатчиков LoRa\texttrademark, 
поскольку 
смягчены требования на точность встроенных кварцевых резонаторов.
Все эти особенности делают возможным для приёмников LoRa\texttrademark~приём 
сигнала 
мощностью 
до -130 дБм.

\subsubsection{Параметры физического уровня} 

Есть несколько параметров для настройки модуляции LoRa\texttrademark: 
\begin{enumerate}
	\item полоса пропускания (\textit{BW}); 
	\item коэффициент расширения спектра (\textit{SF}); 
	\item кодовая скорость (\textit{CR}).
\end{enumerate}

Также могут быть изменены следующие настройки радиомодулей:
\begin{itemize}
	\item длина преамбулы, значение синхронизирующего слова SyncWord;
	\item посылать ли явно заголовок с сообщением, он содержит информацию о 
параметрах приёма остальной части сообщения (длина полезных данных, параметр CR 
и наличие CRC);
	\item наличие поля CRC;
	\item бит LowDataRateOptimization (DE).
\end{itemize}

\subsubsection{Модуляция и кодирование}

\paragraph{Радиосигнал с линейной частотной модуляцией (ЛЧМ)} \hspace{0pt}\\

Физический радиоинтерфейс LoRa\texttrademark~использует широкополосные сигналы 
с линейной 
частотной модуляцией \cite{augustin2016}. 
ЛЧМ ""--- давно известная технология, применявшаяся в радиолокации, но в 
качестве основы кодирования цифровых данных использовалась реже.
При линейной модуляции частота сигнала испытывает линейную девиацию (возрастает 
или уменьшается) со временем. 
Частота изменяется в пределах ширины канала частот (с англ. \textit{Bandwith, 
BW}), таким образом, она изменяется по закону:
\begin{equation}
	\omega(t) = \omega_0 + \mu t
\end{equation}

Здесь $\omega_0$ - несущая частота; $\mu$ - параметр с размерностью с$^{-2}$, 
равный скорости изменения частоты во времени.

За время, равное длительности импульса, девиация частоты равна
\begin{equation}
	\Delta \omega = \mu \tau_{\text{и}},
\end{equation}

где $\tau_{\text{и}}$ ""--- длительность сигнала, а полная фаза сигнала:
\begin{equation}
	\psi(t) = \omega_0 + \mu t{^2}/2
\end{equation}

Сигнал ЛЧМ представляется следующей математической моделью \cite{Baskakov2003}:
\begin{equation}
	u_\text{ЛЧМ}(t) = 
	\begin{cases}
		0, & t < -\tau_\text{и}/2,\\
		U_m cos(\omega_0 t + \mu t{^2}/2), & -\tau_\text{и}/2 \le t \le 
\tau_\text{и}/2,\\
		0, & t > \tau_\text{и}/2.
	\end{cases}
\end{equation}

Анализ характера частотной зависимости модуля и фазы спектральной плотности 
прямоугольного ЛЧМ-импульса выявил \cite{Baskakov2003} полную зависимость от 
безразмерного числа:
\begin{equation}
	B = \Delta f \tau_\text{и} = \mu \tau^{2}_{\text{и}}/(2\pi),
\end{equation}

равного произведению девиации частоты на длительность импульса, называемого 
\textit{базой} ЛЧМ-сигнала.

На практике обычно стараются выполнить условие $B \gg 1$. 
Спектр таких ЛЧМ-сигналов имеет ряд особенностей, и одной из них является то, 
что модуль спектральной плотности практически постоянен в пределах полосы 
частот 
шириной $\Delta \omega$ с центром в точке $\omega_0$.
В LoRa\texttrademark~этим параметром косвенно манипулируют за счёт увеличения 
коэффициента 
расширения спектра (\textit{Spreading Factor}, \textit{SF}).
SF ""--- это логарифмический параметр, соответствующий продолжительности 
передачи одного символа:
\begin{equation}
	B = \Delta f \tau_\text{и} = 2^{SF}
\end{equation}

Значение SF может варьироваться от 6 до 12.

LoRa\texttrademark~кодирует символы циклическим сдвигом ЛЧМ-сигнала 
относительно кадра 
времени. 
Скачок фазы и обозначает кодируемый символ.
Поскольку $2^{SF}$ ЛЧМ-сигнала могут находится в символе, то и один символ 
может эффективно кодировать SF бит информации.

Пропускная способность ЛЧМ-сигналов зависит только от ширины полосы частот.
Увеличение SF повлечет за собой деление продолжительности ЛЧМ-сигнала на два 
(поскольку $2^{SF}$ ЛЧМ-сигналов покрывают всю ширину полосы частот) и 
увеличением в два раза продолжительности передачи символа.
Пропускная способность не уменьшиться в два раза, поскольку теперь каждый 
символ 
кодирует на один бит больше.

Также LoRa\texttrademark~реализован механизм прямой коррекции ошибок 
(\textit{Forward Error 
Correction}, \textit{FEC}).
Параметр CR может быть равен $4/(4+n)$, где $n \in \{1, 2, 3, 4\}$.
Принимая это во внимание, можно вычислить полезную пропускную способность по 
формуле \ref{formula:rb}.
\begin{equation}
 R_b = SF \frac{\Delta f}{2^{SF}} CR \label{formula:rb}
\end{equation}

Для примера, если имеем $\Delta f$ (Он же BW) = 125 кГц, SF = 8, CR = $4/8$, то 
получаем полезную пропускную способность равную 1,95 Кбит/с.
Все вышеприведенные особенности позволяют добиться высокой помехозащищённости 
и, 
как следствие, большой зоны покрытия сети.

Пример сигнала, принятым анализатором спектра от передатчика 
LoRa\texttrademark, изображен на 
рисунке \ref{fig:loradecoding}.

\begin{figure}[!h]
  \centering
  \includegraphics[height=0.4\textheight]{inc/img/LoRaDecoding}
  \caption{Сигнал LoRa\texttrademark~(внизу) и его спектр (вверху) 
\cite{DecodingLora2018}}
  \label{fig:loradecoding}
\end{figure}

\subsubsection{Формат кадра физического уровня}\label{part:physframe}

Хотя модуляция LoRa\texttrademark~позволяет отправлять произвольные кадры, 
формат кадра 
физического уровня был описан и реализован в передатчика и приёмниках от 
Semtech.
Ширина полосы частот и коэффициент расширения спектра не изменяются в рамках 
кадра.

Кадр LoRa\texttrademark~начинается с преамбулы (см. рис. 
\ref{fig:loraphysframe}): 
последовательности растущих ЛЧМ-сигналов, занимающих всю частотную полосу.
Два последних ЛЧМ-сигнала кодируют синхронизирующее слово.
Синхронизирующее слово это одно-байтовое значение, которое используется для 
опознания двух разных сетей, использующих один и тот же канал для связи. 
Устройство, сконфигурированное на приём заданного слова синхронизации остановит 
приём данных если принятое слово синхронизации не совпало с заданным.
Слово синхронизации следует за двумя или четырьмя спадающими ЛЧМ-сигналами в 
преамбуле.
Длина преамбулы может быть изменено между 10,25 и 65539,25 символами.

Когда после преамбулы задаётся необязательный заголовок, он кодируется и 
передаётся с CR = $4/8$.
Он указывает на размер полезной нагрузки (в байтах), CR используемое для 
кодирования и наличие необязательного 16-битного поля CRC.
CRC (если оно есть) находится в конце кадра, сразу после полезной нагрузки.
Поле размера полезной нагрузки имеет размер в один байт, что ограничивает 
максимальный размер полезной нагрузки в 255 байт.

\begin{figure}[!h]
  \centering
  \includegraphics[width=\textwidth]{inc/img/LoRaPhysFrame.pdf}
  \caption{Структура кадра физического уровня LoRa\texttrademark}
  \label{fig:loraphysframe}
\end{figure}

%Поддерживаемые радиоинтерфейсом LoRa\texttrademark~значения BW: 125, 250, 500 
кГц.

\subsubsection{Доступные частотные диапазоны}

Физический уровень LoRa\texttrademark~работает в различных субгигагерцовых 
частотных ISM 
диапазонах.
В различных странах приняты различные диапазоны частот, поэтому устройствам с 
LoRa\texttrademark~необходимо самостоятельно выбирать используемый 
радиодиапазон для 
физического уровня, в зависимости от принятых местных соглашениях. 
Таблица \ref{tab:ismbands} описывает, принятые в различных странах, частотные 
диапазоны для строительства сетей связи.

\begin{table}[ht]
  \caption{частотные диапазоны, принятые в разных странах}
  \begin{tabular}{|l|c|}
  \hline
	  Частотный диапазон & Страна\\
  \hline
	  EU 863-870 МГц ISM & Европейский союз\\
  \hline
	  EU 433 МГц ISM & Европейский союз\\
  \hline
	  RU 864-870 МГц ISM & Российская Федерация\\
  \hline
	  US 902-928 МГц ISM & США\\
  \hline
	  CN 779-787 МГц ISM & КНР\\
  \hline
  \end{tabular}
  \label{tab:ismbands}
\end{table}

В 2017 году был утвержден частотный план LoRa\texttrademark~Alliance, в котором 
определены 
единые региональные параметры LoRaWAN\texttrademark, используемые на территории 
РФ.
Согласно этому плану, для РФ выделяется частотный диапазон шириной в 6 МГц, с 
максимальной шириной полосы частот в 125 кГц.
При этом для конечных устройств \textbf{обязательна} поддержка двух каналов с 
несущими 868,9 и 869,1 МГц/ DR0 до DR5 (см. таблицу \ref{tab:rudefchannels}).

% TODO: Про доступные частотные диапазоны можно написать больше

\begin{table}[ht]
  \caption{Стандартные каналы частотного диапазона RU864-870}
\begin{tabular}{|l|l|l|l|l|l|}
\hline
	Модуляция & \begin{tabular}[c]{@{}l@{}}Ширина \\ полосы\\ частот \\ 
{[}кГц{]}\end{tabular} & \begin{tabular}[c]{@{}l@{}}Несущая\\ 
{[}МГц{]}\end{tabular}& \begin{tabular}[c]{@{}l@{}}Пропускная\\ способность\\ 
FSK или \\ LoRa\texttrademark~DR\end{tabular}& 
\begin{tabular}[c]{@{}l@{}}Кол-во\\ 
каналов\end{tabular} & \begin{tabular}[c]{@{}l@{}}Рабочий\\ цикл\end{tabular} 
\\ 
\hline
	\multicolumn{1}{|c|}{LoRa\texttrademark} & \multicolumn{1}{c|}{125}& 
\multicolumn{1}{c|}{\begin{tabular}[c]{@{}c@{}}868,9\\ 869,1\end{tabular}} & 
\multicolumn{1}{c|}{\begin{tabular}[c]{@{}c@{}}DR0 до DR5 \\ / 0,3-5 
Кбит/c\end{tabular}} & \multicolumn{1}{c|}{2}& \multicolumn{1}{c|}{\textless 
1\%}\\ \hline
\end{tabular}
  \label{tab:rudefchannels}
\end{table}


\paragraph{Радиопомехи} \hspace{0pt}\\

Поскольку в РФ LoRa\texttrademark~использует нелицензируемый диапазон частот, 
то строящиеся 
сети будут работать в условиях внешних помех, создаваемыми прочими 
пользователями диапазона, включая коммерческие и частные сети Интернета вещей, 
работающих по технологиям LoRa\texttrademark, NB-Fi, ``СТРИЖ Телематика'' и пр.

%%%%%%%%%%%%%%%%%%%%%%%%%%%
%%%%%%%%%%%%%%%%%%%%%%%%%%%
\subsection{Протокол LoRaWAN\texttrademark}

LoRaWAN\texttrademark~""--- это MAC протокол, созданный, преимущественно для 
сетей сенсоров \cite{augustin2016, lavric2017internet}, которые обмениваются 
пакетами с сервером на небольшой скорости и на относительно больших интервалах 
времени (одна передача в час или даже в день).
На данном уровне обеспечиваются:
\begin{itemize}
 \item адаптация скорости передачи данных;
 \item шифрование полезной нагрузки на уровне сети, передаваемой между конечным 
устройством и приложением;
 \item управление выделением окон для нисходящего канала связи;
\end{itemize}


\subsubsection{Компоненты сети LoRaWAN\texttrademark}

В спецификации LoRaWAN\texttrademark~определены несколько компонентов для 
создания сети: конечные устройства (\textit{end-devices}), шлюзы (или базовые 
станции) и сетевой сервер (\textit{network server}).
\begin{itemize}
 \item конечное устройство представляет собой, как правило, сенсор с небольшим 
энергопотреблением, которое обменивается данными с базовой станцией, использую 
LoRa\texttrademark;
 \item шлюз: промежуточное устройства, перенаправляющее пакеты, приходящие от 
конечных устройств на сетевой сервер, который имеет соединение с сетью 
Интернет. 
В сети могут находится несколько шлюзов и один и тот же пакет с данными может 
быть получен (и перенаправлен) несколькими шлюзами одновременно;
 \item сетевой сервер: ответственен за устранение повторяющихся пакетов и 
декодирования пакетов, отправленных устройствами и отправки пакетов устройствам.
\end{itemize}

В отличии от традиционных сотовых сетей, конечные устройства не ассоциированы с 
конкретными шлюзами с целью получения доступа к сети.
Шлюзы только предоставляют услуги транспорта пакетов от конечных устройств до 
сетевой сервер включают в пакет информацию о качестве связи. 
Таким образом, конечные устройства соединены с узсетевой сервер который 
ответственен за обнаружение дублирующихся пакетов, выбора подходящего шлюза для 
отправки ответа и т.д.
Логически, шлюзы прозрачны для конечных устройств \cite{augustin2016}.

LoRaWAN\texttrademark~определяет три класса конечных устройств для 
удовлетворения нужд различных приложений:
\begin{itemize}
 \item Класс A, полудуплекс: устройства могут планировать передачу данных по 
восходящему каналу связи (\textit{UL}) в соответствии со своими нуждами. Этот 
класс устройств получают пакеты только после отправки своего пакета в сеть. 
После отправки открываются два небольших окна приёма. Данные по нисходящему 
каналу (\textit{DL}) должны поступить точно во время открытия окон приёма. Эти 
устройства имеют наименьший уровень энергопотребления, но предоставляют меньше 
гибкости для передачи им данных.
 \item Класс B, полудуплекс со запланированными слотами приёма: данный класс 
устройств открывают дополнительные окна приёма данных в назначенное время. Для 
временной синхронизации им требуется маячковый сигнал от шлюзов поблизости. С 
такой синхронизации сетевой сервер знает когда конечное устройство находится в 
состоянии ожидания приёма данных.
 \item Класс C, полудуплекс с постоянным прослушиванием канала: устройства 
данного класса имеют самое протяженное окно приёма и, соответственно, 
наибольшее 
среди остальных потребление энергии.
\end{itemize}

Следует отметить, что LoRaWAN\texttrademark~не допускает коммуникации между 
конечными устройствами: пакеты только могут быть отправлены от конечного 
устройства на сетевой сервер или наоборот. 
Передача данных от одного устройства на другое, если она потребуется, может 
осуществляться только через сетевой сервер (и, соответственно, через все 
промежуточные шлюзы).

\subsubsection{Формат сообщения LoRaWAN\texttrademark} 

LoRaWAN\texttrademark~использует на физическом уровне формат кадра, описанный в 
разделе \ref{part:physframe}.
Заголовок и CRC в сообщениях восходящего канала являются обязательными, что 
делает невозможным использование SF равным шести с сетях LoRaWAN\texttrademark.
Сообщения нисходящего трафика содержат заголовки, но не имеют поля CRC.

Формат сообщения подробно описан на рисунке \ref{fig:macframe}.

\begin{enumerate}
 \item \textit{MHDR} ""--- заголовок пакет уровня MAC. Содержит:
 
 \begin{enumerate}
  \item поле \textit{Major} (2 бита) ""--- определяет major часть версии 
формата 
сообщений процедуры активации по воздуху (OTA - over-the-air);
  \item поле \textit{MType}, определяющее тип сообщения (3 бита). Существует 6 
типов сообщений (см. таблицу \ref{tab:mtypes});
 \end{enumerate}
 
 \item \textit{MACPayload} ""--- фрейм данных. Данный фрейм состоит из 
следующих 
подполей:
 
 \begin{enumerate}
  \item \textit{FHDR} ""--- заголовок фрейма. Он включает в себя:
  
  \begin{enumerate}
   \item \textit{DevAddr} ""--- адрес устройства;
   \item \textit{FCtrl} ""--- октет управляющей информацией фрейма. Состоит из:
   
   \begin{itemize}
    \item \textit{ADR} (1 бит) ""--- флаг режима адаптации скорости;
    \item \textit{ADRAckReq} (1 бит) ""--- флаг, устанавливающийся только в 
режиме адаптации скорости, указывает на запрос устройством подтверждения факта 
получения сообщений от данного устройства;
    \item \textit{FPending} (1 бит, только DL канал) ""--- флаг, обозначающий 
наличие запроса со стороны сети на передачу устройству дополнительных данных 
сверх объема ограничения на максимальный размер кадра;
    \item \textit{CLASS-B} (1 бит, только UL канал) ""--- флаг, обозначающий 
что 
конечное устройство переключилось в режим класса B;
    \item \textit{FOptLen} (4 бита) ""--- размер полня опций FOpt заголовка MAC 
уровня;
   \end{itemize}
   
   \item \textit{FCnt} (16 бит) ""--- номер фрейма. После процедуры активации 
по 
воздуху (OTA), конечное устройство и сетевой сервер инициализируют два счётчика 
""--- счетчик количества принятых фреймов и количества переданных фреймов. При 
получении каждого нового сообщения принимающая сторона сравнивает значение поля 
\textit{FCnt} со значением внутреннего счётчика принятых фреймов. Если разница 
превышает MAX\_FCNT\_GAP, принимается решение о большом количестве потерянных 
фреймов;
   \item \textit{FOpt} (0..120 бит) ""--- опциональные данные (до 15 октетов). 
Используется для передачи команд MAC. Команды MAC могут отправляться в поле 
\textit{FOpt} (и тогда \textit{FOptLen} > 0 и \textit{FPort} > 0), так и в поле 
полезной нагрузки \textit{FRMPayload} (тогда \textit{FOptLen} = 0 и 
\textit{FPort} = 0);
   
  \end{enumerate}
  
  \item \textit{FPort} ""--- номер порта фрейма.
  
  \begin{itemize}
   \item если оно равно 0, это значит что полезная нагрузка содержит MAC 
команду. В этом случае поле FOptLen должно быть равно 0;
   \item значения от 1 до 223 определяются приложением для своих нужд 
(\textit{application specific});
   \item значения 224-225 зарезервированы.
  \end{itemize} 
  
  \item \textit{FRMPayload} ""--- полезная нагрузка. Содержит пользовательские 
данные, которые передаются между целевым приложением и конечным устройством. 
Содержимое этого поля шифруется по стандарту AES либо на сетевом уровне (с 
использованием 128-битного ключа \textit{NwkSKey}), либо на уровне приложения 
(128-битным ключом \textit{AppSKey}).
  
 \end{enumerate}
 
 \item \textit{MIC} (\textit{Message Integrity Code}) "" --- код контроля 
целостности сообщения. Вычисляется алгоритмом AES128 с ключом \textit{NwkSKey} 
по всем полям сообщения.
 
\end{enumerate}


%Счётчиком порядковых номеров кадров служит поле \textit{FCnt}.

\begin{table}[ht]
  \caption{Допустимые значения поля MType}
  \begin{tabular}{|l|l|}
  \hline
  MType & Описание                                                              
 
                                         \\ \hline
  000   & \begin{tabular}[c]{@{}l@{}}Запрос процедуры активации \\ по воздуху 
(OTA) ""--- join request\end{tabular}       \\ \hline
  001   & \begin{tabular}[c]{@{}l@{}}Подтверждение процедуры \\ активации по 
воздуху (OTA) ""--- join accept\end{tabular} \\ \hline
  010   & \begin{tabular}[c]{@{}l@{}}Передача данных “вверх” \\ без 
подтверждения (unconfirmed data up)\end{tabular}      \\ \hline
  011   & \begin{tabular}[c]{@{}l@{}}Передача данных “вниз” \\ без 
подтверждения 
(unconfirmed data down\end{tabular}      \\ \hline
  100   & \begin{tabular}[c]{@{}l@{}}Передача данных “вверх”\\  с 
подтверждением 
(confirmed data up)\end{tabular}         \\ \hline
  101   & \begin{tabular}[c]{@{}l@{}}Передача данных “вниз” \\ с подтверждением 
(confirmed data down)\end{tabular}        \\ \hline
  110   & RFU                                                                   
 
                                         \\ \hline
  111   & Для пользовательских решений                                          
 
                                         \\ \hline
  \end{tabular}
  \label{tab:mtypes}
\end{table}


\begin{figure}[!h]
  \centering
  \includegraphics[width=\textwidth]{inc/img/LoRaMACMsgFmt}
  \caption{Формат кадра LoRaWAN\texttrademark. Размеры полей указаны в битах 
\cite{augustin2016}}
  \label{fig:macframe}
\end{figure}


\subsubsection{Возможные расширения протокола LoRaWAN\texttrademark}

\paragraph{Промышленная адаптация} \hspace{0pt}\\

Для работы протокола LoRaWAN\texttrademark~в концепции индустрии 4.0 
исследователи Mattia 
Rizzi, Paolo Ferrari и др. в своей работе \cite{Rizzi2017} предложили 
адаптировать протокол LoRaWAN\texttrademark, добавив к нему поддержку быстрого 
переключения 
каналов на каждый временной слот(\textit{Time Slotted Channel Hopping, TSCH}).
Исследователями был проведён анализ, в результате которого удалось выяснить, 
что, используя планирование параметрами физического уровня LoRa\texttrademark, 
возможно 
достичь безошибочного опроса конечных устройств интенсивностью до 6000 раз в 
минуту.
Подобная адаптация протокола LoRaWAN\texttrademark~позволит составить ему 
конкуренцию с 
беспроводными технологиями на базе HART (\textit{Highway Addressable Remote 
Transducer Protocol}).

\paragraph{Адаптация к работе в сетях IPv6} \hspace{0pt}\\

Поскольку часть протоколов LPWAN и LoRaWAN\texttrademark, в частности, не 
предполагают 
прямого использования в собственном стеке протокола IP, то процесс интеграции
систем в инфраструктуру Интернета вещей с использованием LPWAN сетей становится 
более сложным, чем должен быть.
Нельзя не упомянуть также о дефиците адресов протокола IPv4, что вновь делает 
актуальным вопрос о массовой интеграции устройств, поддерживающих протокол 
IPv6.

В исследовании Patrick Weber, Axel Sikora и др. было предложено решение по 
адаптации LoRaWAN\texttrademark~к работе в IPv6-сетях, добавлением поддержки 
протокола IPv6 в 
стек протоколов LoRaWAN\texttrademark~\cite{weber2016ipv6}, подобно тому как 
было предложена 
адаптация IPv6 для стандарта IEEE 802.15.4.
Данное решение было названо 6LoRaWAN, а также было реализовано и 
протестировано авторами работы.

Особенностью данной адаптации является то, что помимо топологии ``звезда из 
звёзд'', будет возможно применение других топологий на базе 
LoRaWAN\texttrademark.
Устройства, поддерживающие IPv6, смогут обмениваться сообщениями с любыми 
устройствами в Интернете, что делает их интеграцию в инфраструктуру Интернета 
вещей более лёгкой.

Данная адаптация также поддерживает обратную совместимость (на уровне шлюзов и 
конечных устройств) с конечными устройствами не поддерживающих расширение IPv6.
Более того, подразумевается поддержка 6LoRaWAN существующими 
маршрутизаторами, которые могут образовать сети с любыми конечными 
IP-устройствами (не только на базе LoRaWAN\texttrademark).

В таблице \ref{tab:6loraosi} представлена адаптация 6LoRaWAN как изменение 
стека LoRaWAN на сетевом, транспортном уровне и на уровне приложения по модели 
OSI/RM.

\begin{table}[ht]
  \caption{Классификация протоколов по модели OSI/RM}
  
\begin{tabular}{|l|c|c|}
\hline
                                                                           & 
\multicolumn{1}{l|}{LoRaWAN\texttrademark}             & 
\multicolumn{1}{l|}{6LoRaWAN}                                \\ \hline
\begin{tabular}[c]{@{}l@{}}Уровень\\ приложений\end{tabular}               & 
\begin{tabular}[c]{@{}c@{}}пользовательское\\ приложение\end{tabular} & 
\begin{tabular}[c]{@{}c@{}}например\\ COAP\end{tabular}      \\ \hline
\begin{tabular}[c]{@{}l@{}}Уровень\\ представления\end{tabular}            &    
 
                                                                  &             
 
                                                \\ \hline
\begin{tabular}[c]{@{}l@{}}Сеансовый\\ уровень\end{tabular}                &    
 
                                                                  &             
 
                                                \\ \hline
\begin{tabular}[c]{@{}l@{}}Транспортный\\ уровень\end{tabular}             &    
 
                                                                  & 
\begin{tabular}[c]{@{}c@{}}например\\ UDP\end{tabular}       \\ \hline
\multirow{2}{*}{\begin{tabular}[c]{@{}l@{}}Сетевой\\ уровень\end{tabular}} & 
\multirow{2}{*}{}                                                     & IPv6    
 
                                                    \\ \cline{3-3} 
                                                                           &    
 
                                                                  & 
\begin{tabular}[c]{@{}c@{}}6LoRaWAN\\ адаптация\end{tabular} \\ \hline
\begin{tabular}[c]{@{}l@{}}Канальный\\ уровень\end{tabular}                & 
LoRaMAC                                                               & LoRaMAC 
 
                                                    \\ \hline
\begin{tabular}[c]{@{}l@{}}Физический\\ уровень\end{tabular}               & 
LoRaPHY                                                               & LoRaPHY 
 
                                                    \\ \hline
\end{tabular}

  \label{tab:6loraosi}
\end{table}

%%%%%%%%%%%%%%%%%%%%%%%%%%%%%%%%%%%%%%%%%%%%%%%%%%%%%%%%%%%%%%%%%%%%%%%%%%%%%%%%
%%%%%%%%%%%%%%%%%%%%%%%%%%%%%%%%

\section{Примеры применения технологии LoRa\texttrademark~в рамках концепции 
Интернета вещей}

Существует большое множество областей, где применима концепция Интернета вещей, 
а значит имеет место быть применение технологиям LPWAN и LoRaWAN\texttrademark~в 
частности.
Перечислим некоторые возможные приложения энергоэффективной технологии 
беспроводной передачи данных дальнего радиуса действия:
\begin{itemize}
 \item ``умное'' городское и сельское освещение;
 \item полевые испытания со съёмом показании в режиме реального времени;
 \item транспорт: управление, мониторинг, логистика;
 \item ``умный'' дом, оснащённый различными сенсорами;
 \item сельское хозяйство: автономные комбайны, сенсоры и т.п;
 \item управление энергией;
 \item здравоохранение: легковесные переносные датчики слежения жизненных 
показаний пациента для слежения в реальном времени;
 \item промышленность и производство; 
 \item измерение и учёт потребления электроэнергии в системах АСКУЭ, а также 
расхода воды;
 \item слежение за местоположением в реальном времени. Например для 
отслеживания 
перемещения животных; 
 \item системы безопасности; 
\end{itemize}


\subsection{Умный светильник}

Умные светильники ""--- это то направление, которое уже активно развивается в 
современной индустрии как за рубежом, так и в России.
В Петрозаводске уже сейчас существует как минимум одна компания, 
специализирующаяся на производстве умных светильников, а также всей необходимой 
сетевой инфраструктурой к ним. 
Речь идёт о компании ООО ПК ``Энергосбережение'', которая c 2012 года 
разрабатывает и производит светодиодное осветительное оборудование ""--- 
уличные 
и промышленные светильники.
В 2017 году произошёл качественный скачок в системе мониторинга и управления.
В светильники начали интегрировать модули (см. рис. \ref{fig:uspherum}) с 
поддержкой LPWAN сетей, а именно LoRaWAN\texttrademark, и, тем самым, смогли 
дать 
дополнительный толчок развитию инфраструктуре Интернета вещей на Северо-Западе 
РФ.

\begin{figure}[!h]
  \centering
  \includegraphics[width=0.5\textwidth]{inc/img/Uspherum}
  \caption{Модуль управления светильником c поддержкой сети LoRa\texttrademark, 
разработанный 
в компании ООО ПК ``Энергосбережение'' \cite{isbergsite}}
  \label{fig:uspherum}
\end{figure}

На рисунке \ref{fig:ulite} отображен используемый в компании принцип управления 
``умными'' светильниками. 
Как можно видеть, не только конечные устройства управляют светильниками, но и 
базовые станции наделены такими функциями, что позволяет гибко разместить сеть 
LoRa\texttrademark~на существующих опорах уличного освещения без выделения 
отдельных от опор 
мест под базовые станции.
Сами светильники могут включаться, выключаться и диммироваться (изменять 
освещённость) посредством чат-ботов. 
Чат-бот ""--- это приложение, построенное на базе существующих мессенджеров и 
не 
требующих от пользователей дополнительной установки сторонних приложений для 
оказания услуг.
Всё, что необходимо сделать пользователю для управления имеющимися у него 
светильниками, это:
\begin{enumerate}
 \item установить мессенджер, который поддерживает чат-ботов (например Telegram 
или Viber);
 \item найти чат компании, обслуживающей инфраструктуру умных светильников
 \item пройти быструю авторизацию (этот пункт может быть опущен, если он 
авторизовался ранее);
 \item выбрать светильник или группу светильников в чате;
 \item нажать кнопку ``включить/выключить'' или ``диммирование'';
\end{enumerate}

Команда будет отправлена на соответствующий светильник или на группу 
светильников.

\begin{figure}[!h]
  \centering
  \includegraphics[width=\textwidth]{inc/img/ulite}
  \caption{Схема управления ``умными'' светильниками \cite{isbergsite}}
  \label{fig:ulite}
\end{figure}

\subsection{Умный счётчик}

На современном рынке представлено большое разнообразие счётчиков электрической 
энергии, имеющие различные интерфейсы для снятия показании внешними 
устройствами. 
Наиболее популярные интерфейсы ""--- импульсный выход, RS-232 и RS-485.
Пример такого счётчика приведён на рисунке \ref{fig:mercury}

\begin{figure}[!h]
  \centering
  \includegraphics[width=\textwidth]{inc/img/merc}
  \caption{Счётчик Меркурий 201.5 с импульсным выходом}
  \label{fig:mercury}
\end{figure}

Любое вычислительное устройство (микроконтроллер, ПЛК, FGPA) с поддержкой 
вышеприведенных интерфейсов и с поддержкой сетей LoRaWAN\texttrademark~может 
быть подключено к 
счётчику, а следом, и к системе автоматизированного сбора и учета 
электроэнергии (АСКУЭ).
Всё вышесказанное справедливо не только для счётчиков электроэнергии, но и для 
измерения потребления воды, газа.

Всё это также входит в концепцию Интернета вещей и даже в концепцию 
индустрии 4.0.


%%%%%%%%%%%%%%%%%%%%%%%%%%%%%%%%%%%%%%%%%%%%%%%%%%%%%%%%%%%%%%%%%%%%%%%%%%%%%%%%
%%%%%%%%%%%%%%%%%%%%%%%%%%%%%%%%

\section{Преимущества и недостатки в сравнении с другими технологиями}

На рынке беспроводных технологии представлены десятки различных технологии, 
удовлетворяющих требованиям LPWAN сетей.
Это и не удивительно, поскольку такое большое разнообразие объясняется текущим 
уровнем развития инфраструктуры Интернета вещей: единые стандарты установлены 
не 
были и на рынке наблюдается ожесточённая конкуренция за право стать де-факто 
стандартом, как это было например с USB и FireWire.

Забегая вперёд, стоит отметить, что все перечисленные технологии различаются 
достаточно существенно и их можно выделить в две группы: широкополосные и 
узкополосные.
К широкополосной (\textit{Ultra Wide Band}) относят LoRa\texttrademark.
К узкополосным (\textit{Ultra Narrow Band}) относят Sigfox, ``Стриж'', NB-Fi и 
др.
Рассмотрим другие технологии энергоэффективной передачи данных на большие 
расстояния, претендующие на использование в Интернете вещей.

\subsection{Sigfox}

Sigfox ""--- французская компания, основанная в 2009 году и специализирующаяся 
на создании сетей с низким энергопотреблением, которым необходимо 
продолжительное время передавать небольшие по объёму данные.

Sigfox использует проприетарную технологию беспроводной связи, работающую в ISM 
диапазоне частот: 868 МГц в Европе и 902 МГц в США. 
Их беспроводная технология использует узкополосные каналы шириной в 100 Гц.

В России на нелицензируемый диапазон около 868 МГц в свободном распоряжении 
можно использовать лишь 500 кГц.

Как факт преймущества UNB-систем над UWB-системами можно отнести количество 
каналов, которые могут сосуществовать одновременно. При ширине каналов 
LoRa\texttrademark~в 
125 кГц, нельзя уместить больше трёх каналов на выделенный частотный диапазон. 
Каналов Sigfox можно выделить около 5 тысяч.

Однако системы на базе UNB крайне чувствительны к установке частоты. 
Если брать в расчёт очень качественные кварцевые резонаторы, доступные на 
рынке, то можно получить погрешность в 5 ppm от несущей частоты, что в случае с 
868 МГц даёт погрешность в $\pm 4340$ Гц! При этом не учитывалась погрешность 
резонатора с колебанием температуры внешней среды. 

Утранять данный недостаток были призваны базовые станции Sigfox, которые могли 
видеть сигнал в более широком частотном диапазоне, но, к сожалению, реализовать 
такие же алгоритмы на дешёвых и экономичных конечных устройствах было 
проблематично, поэтому связь Sigfox долгое время была строго однонаправленной.

Системы с UWB обеспечивают симметричный канал связи, благодаря полосе шириной в 
сотни килогерц. 
LoRa\texttrademark~допускает уход от частоты на 25\% от ширины канала, что в 
диапазоне 868 МГц означает допустимую погрешность кварцевого резонатора в 35 
ppm. 
Этой погрешности хватает для того чтобы конечному устройству без проблем 
передавать данные в полном диапазоне температур от -40 до +85 $^\circ$C.

Также UWB-системы не сильно чувствительны к допплеровскому эффекту, что нельзя 
сказать для UNB-систем. 
Sigfox может потерять стабильность работы уже на скоростях в районе 5-10 км/ч.

Скорость передачи данных в UNB-системах фиксирована. И фиксирована она шириной 
полосы частот.
У Sigfox максимальная скорость передачи данных равна 100 бит/с.
UWB-системы могут работать на адаптивной скорости. 
Конечные устройства LoRa\texttrademark~могут перейти в режим ADR 
(\textit{Adaptive Data 
Rate}), позволяя работать на скоростях от 30 бит/с до 50 кбит/с.
Режим ADR меняет скорость передачи автоматически.
Это неизбежно делает UWB-системы более гибкими для применения в разных сферах 
Интернета вещей.

В Sigfox также есть жёсткое ограничение по объёму пользовательских данных ""--- 
один пакет не может передать больше 12 байт, а максимальное количество 
сообщений, которое устройство может передать в сеть, составляет 140, что ставит 
крест на возможном применении данной технологии во многих приложениях Интернета 
вещей.

%\subsection{XNB (Стриж)}

%\subsection{NB-Fi}

%\subsection{LTE/UMTS/GSM}

%Теперь мы покажем, как изменить нумерацию на «нормальную», если вам этого захочется. Пара команд в начале документа поможет нам.

%\renewcommand{\labelenumi}{\arabic{enumi})}
%\renewcommand{\labelenumii}{\asbuk{enumii})}

%\begin{enumerate}
%\item Изменим нумерацию на более привычную...
%\item ... нарушим этим гост.
%\begin{enumerate}
%\item Но, пожалуй, так лучше.
%\end{enumerate}
%\end{enumerate}

%В заключение покажем произвольные маркеры в списках. Для них нужен пакет \textbf{enumerate}.
%\begin{enumerate}[1.]
%\item Маркер с арабской цифрой и с точкой.
%\item Маркер с арабской цифрой и с точкой.
%\begin{enumerate}[I.]
%\item Римская цифра с точкой.
%\item Римская цифра с точкой.
%\end{enumerate}
%\end{enumerate}

%В отчётах могут быть и таблицы "--- см. табл.~\ref{tab:tabular} и~\ref{tab:longtable}.
%Небольшая таблица делается при помощи \Code{tabular} внутри \Code{table} (последний
%полностью аналогичен \Code{figure}, но добавляет другую подпись).

%\begin{table}[ht]
  %\caption{Пример короткой таблицы с коротким названием}
  %\begin{tabular}{|r|c|c|c|l|}
  %\hline
  %Тело      & $F$ & $V$  & $E$ & $F+V-E-2$ \\
  %\hline
  %Тетраэдр  & 4   & 4    & 6   & 0         \\
  %Куб       & 6   & 8    & 12  & 0         \\
  %Октаэдр   & 8   & 6    & 12  & 0         \\
  %Додекаэдр & 20  & 12   & 30  & 0         \\
  %Икосаэдр  & 12  & 20   & 30  & 0         \\
  %\hline
  %Эйлер     & 666 & 9000 & 42  & $+\infty$ \\
  %\hline
  %\end{tabular}
  %\label{tab:tabular}
%\end{table}

%Для больших таблиц следует использовать пакет \Code{longtable}, позволяющий создавать
%таблицы на несколько страниц по ГОСТ.

%Для того, чтобы длинный текст разбивался на много строк в пределах одной ячейки, надо в
%качестве ее формата задавать \texttt{p} и указывать явно ширину: в мм/дюймах
%(\texttt{110mm}), относительно ширины страницы (\texttt{0.22\textbackslash textwidth})
%и~т.п.

%Можно также использовать уменьшенный шрифт "--- но, пожалуйста, тогда уж во \textbf{всей}
%таблице сразу.

%\begin{center}
  %\begin{longtable}{|p{0.40\textwidth}|c|p{0.30\textwidth}|}
    %\caption{Пример длинной таблицы с длинным названием на много длинных-длинных строк}
    %\label{tab:longtable}
    %\\ \hline
    %Вид шума & Громкость, дБ & Комментарий \\
    %\hline \endfirsthead
    %\subcaption{Продолжение таблицы~\ref{tab:longtable}}
    %\\ \hline \endhead
    %\hline \subcaption{Продолжение на след. стр.}
    %\endfoot
    %\hline \endlastfoot
    %Порог слышимости             & 0     &                                                \\
    %\hline
    %Шепот в тихой библиотеке     & 30    &                                                \\
    %Обычный разговор             & 60-70 &                                                \\
    %Звонок телефона              & 80    & \small{Конечно, это было до эпохи мобильников} \\
    %Уличный шум                  & 85    & \small{(внутри машины)}                        \\
    %Гудок поезда                 & 90    &                                                \\
    %Шум электрички               & 95    &                                                \\
    %\hline
    %Порог здоровой нормы         & 90-95 & \small{Длительное пребывание на более
    %громком шуме может привести к ухудшению слуха}                                        \\
    %\hline
    %Мотоцикл                     & 100   &                                                \\
    %Power Mower                  & 107   & \small{(модель бензокосилки)}                  \\
    %Бензопила                    & 110   & \small{(Doom в целом вреден для здоровья)}     \\
    %Рок-концерт                  & 115   &                                                \\
    %\hline
    %Порог боли                   & 125   & \small{feel the pain}                          \\
    %\hline
    %Клепальный молоток           & 125   & \small{(автор сам не знает, что это)}          \\
    %\hline
    %Порог опасности              & 140   & \small{Даже кратковременное пребывание на
    %шуме большего уровня может привести к необратимым последствиям}                       \\
    %\hline
    %Реактивный двигатель         & 140   &                                                \\
                                 %& 180   & \small{Необратимое полное повреждение
                                 %слуховых органов}                                        \\
    %Самый громкий возможный звук & 194   & \small{Интересно, почему?..}                   \\
  %\end{longtable}
%\end{center}

%%% Local Variables:
%%% mode: latex
%%% TeX-master: "rpz"
%%% End:
